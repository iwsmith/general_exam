
% Default to the notebook output style

    


% Inherit from the specified cell style.




    
\documentclass[11pt]{article}

    
    
    \usepackage[T1]{fontenc}
    % Nicer default font (+ math font) than Computer Modern for most use cases
    \usepackage{mathpazo}

    % Basic figure setup, for now with no caption control since it's done
    % automatically by Pandoc (which extracts ![](path) syntax from Markdown).
    \usepackage{graphicx}
    % We will generate all images so they have a width \maxwidth. This means
    % that they will get their normal width if they fit onto the page, but
    % are scaled down if they would overflow the margins.
    \makeatletter
    \def\maxwidth{\ifdim\Gin@nat@width>\linewidth\linewidth
    \else\Gin@nat@width\fi}
    \makeatother
    \let\Oldincludegraphics\includegraphics
    % Set max figure width to be 80% of text width, for now hardcoded.
    \renewcommand{\includegraphics}[1]{\Oldincludegraphics[width=.8\maxwidth]{#1}}
    % Ensure that by default, figures have no caption (until we provide a
    % proper Figure object with a Caption API and a way to capture that
    % in the conversion process - todo).
    \usepackage{caption}
    \DeclareCaptionLabelFormat{nolabel}{}
    \captionsetup{labelformat=nolabel}

    \usepackage{adjustbox} % Used to constrain images to a maximum size 
    \usepackage{xcolor} % Allow colors to be defined
    \usepackage{enumerate} % Needed for markdown enumerations to work
    \usepackage{geometry} % Used to adjust the document margins
    \usepackage{amsmath} % Equations
    \usepackage{amssymb} % Equations
    \usepackage{textcomp} % defines textquotesingle
    % Hack from http://tex.stackexchange.com/a/47451/13684:
    \AtBeginDocument{%
        \def\PYZsq{\textquotesingle}% Upright quotes in Pygmentized code
    }
    \usepackage{upquote} % Upright quotes for verbatim code
    \usepackage{eurosym} % defines \euro
    \usepackage[mathletters]{ucs} % Extended unicode (utf-8) support
    \usepackage[utf8x]{inputenc} % Allow utf-8 characters in the tex document
    \usepackage{fancyvrb} % verbatim replacement that allows latex
    \usepackage{grffile} % extends the file name processing of package graphics 
                         % to support a larger range 
    % The hyperref package gives us a pdf with properly built
    % internal navigation ('pdf bookmarks' for the table of contents,
    % internal cross-reference links, web links for URLs, etc.)
    \usepackage{hyperref}
    \usepackage{longtable} % longtable support required by pandoc >1.10
    \usepackage{booktabs}  % table support for pandoc > 1.12.2
    \usepackage[inline]{enumitem} % IRkernel/repr support (it uses the enumerate* environment)
    \usepackage[normalem]{ulem} % ulem is needed to support strikethroughs (\sout)
                                % normalem makes italics be italics, not underlines
    

    
    
    % Colors for the hyperref package
    \definecolor{urlcolor}{rgb}{0,.145,.698}
    \definecolor{linkcolor}{rgb}{.71,0.21,0.01}
    \definecolor{citecolor}{rgb}{.12,.54,.11}

    % ANSI colors
    \definecolor{ansi-black}{HTML}{3E424D}
    \definecolor{ansi-black-intense}{HTML}{282C36}
    \definecolor{ansi-red}{HTML}{E75C58}
    \definecolor{ansi-red-intense}{HTML}{B22B31}
    \definecolor{ansi-green}{HTML}{00A250}
    \definecolor{ansi-green-intense}{HTML}{007427}
    \definecolor{ansi-yellow}{HTML}{DDB62B}
    \definecolor{ansi-yellow-intense}{HTML}{B27D12}
    \definecolor{ansi-blue}{HTML}{208FFB}
    \definecolor{ansi-blue-intense}{HTML}{0065CA}
    \definecolor{ansi-magenta}{HTML}{D160C4}
    \definecolor{ansi-magenta-intense}{HTML}{A03196}
    \definecolor{ansi-cyan}{HTML}{60C6C8}
    \definecolor{ansi-cyan-intense}{HTML}{258F8F}
    \definecolor{ansi-white}{HTML}{C5C1B4}
    \definecolor{ansi-white-intense}{HTML}{A1A6B2}

    % commands and environments needed by pandoc snippets
    % extracted from the output of `pandoc -s`
    \providecommand{\tightlist}{%
      \setlength{\itemsep}{0pt}\setlength{\parskip}{0pt}}
    \DefineVerbatimEnvironment{Highlighting}{Verbatim}{commandchars=\\\{\}}
    % Add ',fontsize=\small' for more characters per line
    \newenvironment{Shaded}{}{}
    \newcommand{\KeywordTok}[1]{\textcolor[rgb]{0.00,0.44,0.13}{\textbf{{#1}}}}
    \newcommand{\DataTypeTok}[1]{\textcolor[rgb]{0.56,0.13,0.00}{{#1}}}
    \newcommand{\DecValTok}[1]{\textcolor[rgb]{0.25,0.63,0.44}{{#1}}}
    \newcommand{\BaseNTok}[1]{\textcolor[rgb]{0.25,0.63,0.44}{{#1}}}
    \newcommand{\FloatTok}[1]{\textcolor[rgb]{0.25,0.63,0.44}{{#1}}}
    \newcommand{\CharTok}[1]{\textcolor[rgb]{0.25,0.44,0.63}{{#1}}}
    \newcommand{\StringTok}[1]{\textcolor[rgb]{0.25,0.44,0.63}{{#1}}}
    \newcommand{\CommentTok}[1]{\textcolor[rgb]{0.38,0.63,0.69}{\textit{{#1}}}}
    \newcommand{\OtherTok}[1]{\textcolor[rgb]{0.00,0.44,0.13}{{#1}}}
    \newcommand{\AlertTok}[1]{\textcolor[rgb]{1.00,0.00,0.00}{\textbf{{#1}}}}
    \newcommand{\FunctionTok}[1]{\textcolor[rgb]{0.02,0.16,0.49}{{#1}}}
    \newcommand{\RegionMarkerTok}[1]{{#1}}
    \newcommand{\ErrorTok}[1]{\textcolor[rgb]{1.00,0.00,0.00}{\textbf{{#1}}}}
    \newcommand{\NormalTok}[1]{{#1}}
    
    % Additional commands for more recent versions of Pandoc
    \newcommand{\ConstantTok}[1]{\textcolor[rgb]{0.53,0.00,0.00}{{#1}}}
    \newcommand{\SpecialCharTok}[1]{\textcolor[rgb]{0.25,0.44,0.63}{{#1}}}
    \newcommand{\VerbatimStringTok}[1]{\textcolor[rgb]{0.25,0.44,0.63}{{#1}}}
    \newcommand{\SpecialStringTok}[1]{\textcolor[rgb]{0.73,0.40,0.53}{{#1}}}
    \newcommand{\ImportTok}[1]{{#1}}
    \newcommand{\DocumentationTok}[1]{\textcolor[rgb]{0.73,0.13,0.13}{\textit{{#1}}}}
    \newcommand{\AnnotationTok}[1]{\textcolor[rgb]{0.38,0.63,0.69}{\textbf{\textit{{#1}}}}}
    \newcommand{\CommentVarTok}[1]{\textcolor[rgb]{0.38,0.63,0.69}{\textbf{\textit{{#1}}}}}
    \newcommand{\VariableTok}[1]{\textcolor[rgb]{0.10,0.09,0.49}{{#1}}}
    \newcommand{\ControlFlowTok}[1]{\textcolor[rgb]{0.00,0.44,0.13}{\textbf{{#1}}}}
    \newcommand{\OperatorTok}[1]{\textcolor[rgb]{0.40,0.40,0.40}{{#1}}}
    \newcommand{\BuiltInTok}[1]{{#1}}
    \newcommand{\ExtensionTok}[1]{{#1}}
    \newcommand{\PreprocessorTok}[1]{\textcolor[rgb]{0.74,0.48,0.00}{{#1}}}
    \newcommand{\AttributeTok}[1]{\textcolor[rgb]{0.49,0.56,0.16}{{#1}}}
    \newcommand{\InformationTok}[1]{\textcolor[rgb]{0.38,0.63,0.69}{\textbf{\textit{{#1}}}}}
    \newcommand{\WarningTok}[1]{\textcolor[rgb]{0.38,0.63,0.69}{\textbf{\textit{{#1}}}}}
    
    
    % Define a nice break command that doesn't care if a line doesn't already
    % exist.
    \def\br{\hspace*{\fill} \\* }
    % Math Jax compatability definitions
    \def\gt{>}
    \def\lt{<}
    % Document parameters
    \title{analysis}
    
    
    

    % Pygments definitions
    
\makeatletter
\def\PY@reset{\let\PY@it=\relax \let\PY@bf=\relax%
    \let\PY@ul=\relax \let\PY@tc=\relax%
    \let\PY@bc=\relax \let\PY@ff=\relax}
\def\PY@tok#1{\csname PY@tok@#1\endcsname}
\def\PY@toks#1+{\ifx\relax#1\empty\else%
    \PY@tok{#1}\expandafter\PY@toks\fi}
\def\PY@do#1{\PY@bc{\PY@tc{\PY@ul{%
    \PY@it{\PY@bf{\PY@ff{#1}}}}}}}
\def\PY#1#2{\PY@reset\PY@toks#1+\relax+\PY@do{#2}}

\expandafter\def\csname PY@tok@sr\endcsname{\def\PY@tc##1{\textcolor[rgb]{0.73,0.40,0.53}{##1}}}
\expandafter\def\csname PY@tok@nf\endcsname{\def\PY@tc##1{\textcolor[rgb]{0.00,0.00,1.00}{##1}}}
\expandafter\def\csname PY@tok@sd\endcsname{\let\PY@it=\textit\def\PY@tc##1{\textcolor[rgb]{0.73,0.13,0.13}{##1}}}
\expandafter\def\csname PY@tok@vg\endcsname{\def\PY@tc##1{\textcolor[rgb]{0.10,0.09,0.49}{##1}}}
\expandafter\def\csname PY@tok@mi\endcsname{\def\PY@tc##1{\textcolor[rgb]{0.40,0.40,0.40}{##1}}}
\expandafter\def\csname PY@tok@ow\endcsname{\let\PY@bf=\textbf\def\PY@tc##1{\textcolor[rgb]{0.67,0.13,1.00}{##1}}}
\expandafter\def\csname PY@tok@fm\endcsname{\def\PY@tc##1{\textcolor[rgb]{0.00,0.00,1.00}{##1}}}
\expandafter\def\csname PY@tok@vi\endcsname{\def\PY@tc##1{\textcolor[rgb]{0.10,0.09,0.49}{##1}}}
\expandafter\def\csname PY@tok@sh\endcsname{\def\PY@tc##1{\textcolor[rgb]{0.73,0.13,0.13}{##1}}}
\expandafter\def\csname PY@tok@kr\endcsname{\let\PY@bf=\textbf\def\PY@tc##1{\textcolor[rgb]{0.00,0.50,0.00}{##1}}}
\expandafter\def\csname PY@tok@no\endcsname{\def\PY@tc##1{\textcolor[rgb]{0.53,0.00,0.00}{##1}}}
\expandafter\def\csname PY@tok@si\endcsname{\let\PY@bf=\textbf\def\PY@tc##1{\textcolor[rgb]{0.73,0.40,0.53}{##1}}}
\expandafter\def\csname PY@tok@gu\endcsname{\let\PY@bf=\textbf\def\PY@tc##1{\textcolor[rgb]{0.50,0.00,0.50}{##1}}}
\expandafter\def\csname PY@tok@kn\endcsname{\let\PY@bf=\textbf\def\PY@tc##1{\textcolor[rgb]{0.00,0.50,0.00}{##1}}}
\expandafter\def\csname PY@tok@sb\endcsname{\def\PY@tc##1{\textcolor[rgb]{0.73,0.13,0.13}{##1}}}
\expandafter\def\csname PY@tok@err\endcsname{\def\PY@bc##1{\setlength{\fboxsep}{0pt}\fcolorbox[rgb]{1.00,0.00,0.00}{1,1,1}{\strut ##1}}}
\expandafter\def\csname PY@tok@ne\endcsname{\let\PY@bf=\textbf\def\PY@tc##1{\textcolor[rgb]{0.82,0.25,0.23}{##1}}}
\expandafter\def\csname PY@tok@gt\endcsname{\def\PY@tc##1{\textcolor[rgb]{0.00,0.27,0.87}{##1}}}
\expandafter\def\csname PY@tok@mo\endcsname{\def\PY@tc##1{\textcolor[rgb]{0.40,0.40,0.40}{##1}}}
\expandafter\def\csname PY@tok@gr\endcsname{\def\PY@tc##1{\textcolor[rgb]{1.00,0.00,0.00}{##1}}}
\expandafter\def\csname PY@tok@il\endcsname{\def\PY@tc##1{\textcolor[rgb]{0.40,0.40,0.40}{##1}}}
\expandafter\def\csname PY@tok@nb\endcsname{\def\PY@tc##1{\textcolor[rgb]{0.00,0.50,0.00}{##1}}}
\expandafter\def\csname PY@tok@nd\endcsname{\def\PY@tc##1{\textcolor[rgb]{0.67,0.13,1.00}{##1}}}
\expandafter\def\csname PY@tok@sx\endcsname{\def\PY@tc##1{\textcolor[rgb]{0.00,0.50,0.00}{##1}}}
\expandafter\def\csname PY@tok@gp\endcsname{\let\PY@bf=\textbf\def\PY@tc##1{\textcolor[rgb]{0.00,0.00,0.50}{##1}}}
\expandafter\def\csname PY@tok@go\endcsname{\def\PY@tc##1{\textcolor[rgb]{0.53,0.53,0.53}{##1}}}
\expandafter\def\csname PY@tok@kd\endcsname{\let\PY@bf=\textbf\def\PY@tc##1{\textcolor[rgb]{0.00,0.50,0.00}{##1}}}
\expandafter\def\csname PY@tok@gs\endcsname{\let\PY@bf=\textbf}
\expandafter\def\csname PY@tok@w\endcsname{\def\PY@tc##1{\textcolor[rgb]{0.73,0.73,0.73}{##1}}}
\expandafter\def\csname PY@tok@ss\endcsname{\def\PY@tc##1{\textcolor[rgb]{0.10,0.09,0.49}{##1}}}
\expandafter\def\csname PY@tok@nl\endcsname{\def\PY@tc##1{\textcolor[rgb]{0.63,0.63,0.00}{##1}}}
\expandafter\def\csname PY@tok@se\endcsname{\let\PY@bf=\textbf\def\PY@tc##1{\textcolor[rgb]{0.73,0.40,0.13}{##1}}}
\expandafter\def\csname PY@tok@s1\endcsname{\def\PY@tc##1{\textcolor[rgb]{0.73,0.13,0.13}{##1}}}
\expandafter\def\csname PY@tok@cp\endcsname{\def\PY@tc##1{\textcolor[rgb]{0.74,0.48,0.00}{##1}}}
\expandafter\def\csname PY@tok@c1\endcsname{\let\PY@it=\textit\def\PY@tc##1{\textcolor[rgb]{0.25,0.50,0.50}{##1}}}
\expandafter\def\csname PY@tok@ch\endcsname{\let\PY@it=\textit\def\PY@tc##1{\textcolor[rgb]{0.25,0.50,0.50}{##1}}}
\expandafter\def\csname PY@tok@s\endcsname{\def\PY@tc##1{\textcolor[rgb]{0.73,0.13,0.13}{##1}}}
\expandafter\def\csname PY@tok@bp\endcsname{\def\PY@tc##1{\textcolor[rgb]{0.00,0.50,0.00}{##1}}}
\expandafter\def\csname PY@tok@cpf\endcsname{\let\PY@it=\textit\def\PY@tc##1{\textcolor[rgb]{0.25,0.50,0.50}{##1}}}
\expandafter\def\csname PY@tok@mb\endcsname{\def\PY@tc##1{\textcolor[rgb]{0.40,0.40,0.40}{##1}}}
\expandafter\def\csname PY@tok@vm\endcsname{\def\PY@tc##1{\textcolor[rgb]{0.10,0.09,0.49}{##1}}}
\expandafter\def\csname PY@tok@sc\endcsname{\def\PY@tc##1{\textcolor[rgb]{0.73,0.13,0.13}{##1}}}
\expandafter\def\csname PY@tok@vc\endcsname{\def\PY@tc##1{\textcolor[rgb]{0.10,0.09,0.49}{##1}}}
\expandafter\def\csname PY@tok@ge\endcsname{\let\PY@it=\textit}
\expandafter\def\csname PY@tok@gi\endcsname{\def\PY@tc##1{\textcolor[rgb]{0.00,0.63,0.00}{##1}}}
\expandafter\def\csname PY@tok@gd\endcsname{\def\PY@tc##1{\textcolor[rgb]{0.63,0.00,0.00}{##1}}}
\expandafter\def\csname PY@tok@mh\endcsname{\def\PY@tc##1{\textcolor[rgb]{0.40,0.40,0.40}{##1}}}
\expandafter\def\csname PY@tok@na\endcsname{\def\PY@tc##1{\textcolor[rgb]{0.49,0.56,0.16}{##1}}}
\expandafter\def\csname PY@tok@cm\endcsname{\let\PY@it=\textit\def\PY@tc##1{\textcolor[rgb]{0.25,0.50,0.50}{##1}}}
\expandafter\def\csname PY@tok@nt\endcsname{\let\PY@bf=\textbf\def\PY@tc##1{\textcolor[rgb]{0.00,0.50,0.00}{##1}}}
\expandafter\def\csname PY@tok@s2\endcsname{\def\PY@tc##1{\textcolor[rgb]{0.73,0.13,0.13}{##1}}}
\expandafter\def\csname PY@tok@gh\endcsname{\let\PY@bf=\textbf\def\PY@tc##1{\textcolor[rgb]{0.00,0.00,0.50}{##1}}}
\expandafter\def\csname PY@tok@kc\endcsname{\let\PY@bf=\textbf\def\PY@tc##1{\textcolor[rgb]{0.00,0.50,0.00}{##1}}}
\expandafter\def\csname PY@tok@sa\endcsname{\def\PY@tc##1{\textcolor[rgb]{0.73,0.13,0.13}{##1}}}
\expandafter\def\csname PY@tok@ni\endcsname{\let\PY@bf=\textbf\def\PY@tc##1{\textcolor[rgb]{0.60,0.60,0.60}{##1}}}
\expandafter\def\csname PY@tok@cs\endcsname{\let\PY@it=\textit\def\PY@tc##1{\textcolor[rgb]{0.25,0.50,0.50}{##1}}}
\expandafter\def\csname PY@tok@dl\endcsname{\def\PY@tc##1{\textcolor[rgb]{0.73,0.13,0.13}{##1}}}
\expandafter\def\csname PY@tok@nn\endcsname{\let\PY@bf=\textbf\def\PY@tc##1{\textcolor[rgb]{0.00,0.00,1.00}{##1}}}
\expandafter\def\csname PY@tok@kp\endcsname{\def\PY@tc##1{\textcolor[rgb]{0.00,0.50,0.00}{##1}}}
\expandafter\def\csname PY@tok@c\endcsname{\let\PY@it=\textit\def\PY@tc##1{\textcolor[rgb]{0.25,0.50,0.50}{##1}}}
\expandafter\def\csname PY@tok@k\endcsname{\let\PY@bf=\textbf\def\PY@tc##1{\textcolor[rgb]{0.00,0.50,0.00}{##1}}}
\expandafter\def\csname PY@tok@nv\endcsname{\def\PY@tc##1{\textcolor[rgb]{0.10,0.09,0.49}{##1}}}
\expandafter\def\csname PY@tok@nc\endcsname{\let\PY@bf=\textbf\def\PY@tc##1{\textcolor[rgb]{0.00,0.00,1.00}{##1}}}
\expandafter\def\csname PY@tok@m\endcsname{\def\PY@tc##1{\textcolor[rgb]{0.40,0.40,0.40}{##1}}}
\expandafter\def\csname PY@tok@o\endcsname{\def\PY@tc##1{\textcolor[rgb]{0.40,0.40,0.40}{##1}}}
\expandafter\def\csname PY@tok@mf\endcsname{\def\PY@tc##1{\textcolor[rgb]{0.40,0.40,0.40}{##1}}}
\expandafter\def\csname PY@tok@kt\endcsname{\def\PY@tc##1{\textcolor[rgb]{0.69,0.00,0.25}{##1}}}

\def\PYZbs{\char`\\}
\def\PYZus{\char`\_}
\def\PYZob{\char`\{}
\def\PYZcb{\char`\}}
\def\PYZca{\char`\^}
\def\PYZam{\char`\&}
\def\PYZlt{\char`\<}
\def\PYZgt{\char`\>}
\def\PYZsh{\char`\#}
\def\PYZpc{\char`\%}
\def\PYZdl{\char`\$}
\def\PYZhy{\char`\-}
\def\PYZsq{\char`\'}
\def\PYZdq{\char`\"}
\def\PYZti{\char`\~}
% for compatibility with earlier versions
\def\PYZat{@}
\def\PYZlb{[}
\def\PYZrb{]}
\makeatother


    % Exact colors from NB
    \definecolor{incolor}{rgb}{0.0, 0.0, 0.5}
    \definecolor{outcolor}{rgb}{0.545, 0.0, 0.0}



    
    % Prevent overflowing lines due to hard-to-break entities
    \sloppy 
    % Setup hyperref package
    \hypersetup{
      breaklinks=true,  % so long urls are correctly broken across lines
      colorlinks=true,
      urlcolor=urlcolor,
      linkcolor=linkcolor,
      citecolor=citecolor,
      }
    % Slightly bigger margins than the latex defaults
    
    \geometry{verbose,tmargin=1in,bmargin=1in,lmargin=1in,rmargin=1in}
    
    

    \begin{document}
    
    
    \maketitle
    
    

    
    \hypertarget{overview}{%
\subsection{Overview}\label{overview}}

This document analyzes the \href{http://dblp.uni-trier.de/}{DBLP}
computer science bibliography, distributed by
\href{https://aminer.org/citation}{AMiner}. I specifically analyze the
\href{https://static.aminer.org/lab-datasets/citation/dblp.v10.zip}{DBLP-Citation-network
V10} dataset, containing 3,079,007 papers and 25,166,994 citations.
Pre-processing of the data was performed in
\href{https://github.com/iwsmith/general_exam/blob/master/preprocessing.py.ipynb}{preprocessing.py}.

We are investigating the following research question: \textgreater{} As
noted in (4) search engines and recommender systems could be affecting
what is being searched in the scholarly literature. Are readers,
collectively, diverging in their reading interests or are they
converging? What evidences supports these alternative interpretations?
Using open, bibliographic data sets, investigate this question. The code
and responses should be written in a Jupypter notebook with text, graphs
and metrics interspersed. Be sure to justify your metrics and analyses
and to include your interpretations of your data.

    \begin{Verbatim}[commandchars=\\\{\}]
{\color{incolor}In [{\color{incolor}35}]:} \PY{o}{\PYZpc{}}\PY{k}{matplotlib} inline
         \PY{k+kn}{import} \PY{n+nn}{pandas} \PY{k}{as} \PY{n+nn}{pd}
         \PY{k+kn}{import} \PY{n+nn}{numpy} \PY{k}{as} \PY{n+nn}{np}
         \PY{k+kn}{import} \PY{n+nn}{seaborn} \PY{k}{as} \PY{n+nn}{sns}
         \PY{n}{figsize}\PY{o}{=}\PY{p}{(}\PY{l+m+mi}{14}\PY{p}{,}\PY{l+m+mi}{8}\PY{p}{)}
\end{Verbatim}

    \begin{Verbatim}[commandchars=\\\{\}]
{\color{incolor}In [{\color{incolor}48}]:} \PY{n}{df}\PY{o}{=}\PY{n}{pd}\PY{o}{.}\PY{n}{read\PYZus{}pickle}\PY{p}{(}\PY{l+s+s2}{\PYZdq{}}\PY{l+s+s2}{data.pickle}\PY{l+s+s2}{\PYZdq{}}\PY{p}{)}
\end{Verbatim}

    \begin{Verbatim}[commandchars=\\\{\}]
{\color{incolor}In [{\color{incolor}3}]:} \PY{n}{df}\PY{o}{.}\PY{n}{head}\PY{p}{(}\PY{p}{)}
\end{Verbatim}

            \begin{Verbatim}[commandchars=\\\{\}]
{\color{outcolor}Out[{\color{outcolor}3}]:}                                       year  \textbackslash{}
        id                                           
        00127ee2-cb05-48ce-bc49-9de556b93346  2013   
        001c58d3-26ad-46b3-ab3a-c1e557d16821  2011   
        001c8744-73c4-4b04-9364-22d31a10dbf1  2009   
        00338203-9eb3-40c5-9f31-cbac73a519ec  2011   
        0040b022-1472-4f70-a753-74832df65266  1998   
        
                                                                                     references  \textbackslash{}
        id                                                                                        
        00127ee2-cb05-48ce-bc49-9de556b93346  [51c7e02e-f5ed-431a-8cf5-f761f266d4be, 69b625b{\ldots}   
        001c58d3-26ad-46b3-ab3a-c1e557d16821  [10482dd3-4642-4193-842f-85f3b70fcf65, 3133714{\ldots}   
        001c8744-73c4-4b04-9364-22d31a10dbf1  [2d84c0f2-e656-4ce7-b018-90eda1c132fe, a083a1b{\ldots}   
        00338203-9eb3-40c5-9f31-cbac73a519ec  [8c78e4b0-632b-4293-b491-85b1976675e6, 9cdc54f{\ldots}   
        0040b022-1472-4f70-a753-74832df65266                                               None   
        
                                                                                        authors  \textbackslash{}
        id                                                                                        
        00127ee2-cb05-48ce-bc49-9de556b93346  [Makoto Satoh, Ryo Muramatsu, Mizue Kayama, Ka{\ldots}   
        001c58d3-26ad-46b3-ab3a-c1e557d16821                        [Gareth Beale, Graeme Earl]   
        001c8744-73c4-4b04-9364-22d31a10dbf1  [Altaf Hossain, Faisal Zaman, Mohammed Nasser,{\ldots}   
        00338203-9eb3-40c5-9f31-cbac73a519ec  [Jea-Bum Park, Byungmok Kim, Jian Shen, Sun-Yo{\ldots}   
        0040b022-1472-4f70-a753-74832df65266                [Giovanna Guerrini, Isabella Merlo]   
        
                                                                                          title  \textbackslash{}
        id                                                                                        
        00127ee2-cb05-48ce-bc49-9de556b93346  Preliminary Design of a Network Protocol Learn{\ldots}   
        001c58d3-26ad-46b3-ab3a-c1e557d16821  A methodology for the physically accurate visu{\ldots}   
        001c8744-73c4-4b04-9364-22d31a10dbf1  Comparison of GARCH, Neural Network and Suppor{\ldots}   
        00338203-9eb3-40c5-9f31-cbac73a519ec  Development of Remote Monitoring and Control D{\ldots}   
        0040b022-1472-4f70-a753-74832df65266  Reasonig about Set-Oriented Methods in Object {\ldots}   
        
                                              out\_cite  author\_cnt  title\_len  \textbackslash{}
        id                                                                      
        00127ee2-cb05-48ce-bc49-9de556b93346         2           8    3079007   
        001c58d3-26ad-46b3-ab3a-c1e557d16821        13           2    3079007   
        001c8744-73c4-4b04-9364-22d31a10dbf1         2           4    3079007   
        00338203-9eb3-40c5-9f31-cbac73a519ec         2           5    3079007   
        0040b022-1472-4f70-a753-74832df65266         0           2    3079007   
        
                                                      flow  in\_cite  
        id                                                           
        00127ee2-cb05-48ce-bc49-9de556b93346  0.000000e+00      NaN  
        001c58d3-26ad-46b3-ab3a-c1e557d16821  2.061690e-08      1.0  
        001c8744-73c4-4b04-9364-22d31a10dbf1  4.608480e-08      2.0  
        00338203-9eb3-40c5-9f31-cbac73a519ec  0.000000e+00      NaN  
        0040b022-1472-4f70-a753-74832df65266           NaN      NaN  
\end{Verbatim}
        
    \begin{Verbatim}[commandchars=\\\{\}]
{\color{incolor}In [{\color{incolor}50}]:} \PY{n}{df}\PY{o}{.}\PY{n}{describe}\PY{p}{(}\PY{p}{)}
\end{Verbatim}

            \begin{Verbatim}[commandchars=\\\{\}]
{\color{outcolor}Out[{\color{outcolor}50}]:}                year      out\_cite    author\_cnt     title\_len          flow  \textbackslash{}
         count  3.079007e+06  3.079007e+06  3.079007e+06  3.079007e+06  2.725533e+06   
         mean   2.007767e+03  8.173737e+00  3.077669e+00  7.316575e+01  3.669007e-07   
         std    7.816538e+00  9.707329e+00  1.780460e+00  2.559721e+01  2.312681e-06   
         min    1.936000e+03  0.000000e+00  0.000000e+00  4.000000e+00  0.000000e+00   
         25\%    2.004000e+03  1.000000e+00  2.000000e+00  5.500000e+01  0.000000e+00   
         50\%    2.010000e+03  6.000000e+00  3.000000e+00  7.100000e+01  5.301480e-08   
         75\%    2.013000e+03  1.200000e+01  4.000000e+00  8.800000e+01  2.095020e-07   
         max    2.018000e+03  1.532000e+03  3.510000e+02  4.760000e+02  7.349310e-04   
         
                     in\_cite  
         count  1.985921e+06  
         mean   1.267271e+01  
         std    5.616878e+01  
         min    1.000000e+00  
         25\%    2.000000e+00  
         50\%    4.000000e+00  
         75\%    1.000000e+01  
         max    1.622900e+04  
\end{Verbatim}
        
    The dataframe consists of 6 columns. Most are self explanitory, except
for \texttt{flow}, which referes to the Eigenfactor score of a given
paper. These scores were generated by running:
\texttt{./infomap\ dblp.pajek\ output\ -t\ -\/-inner-parallelization\ -i\ pjk}

Of note is that the collection spans from 1936 to 2018, with a maximum
of 1,523 out citations (``Image analysis and computer vision : 1998''),
and 16,229 in citations (``Distinctive Image Features from
Scale-Invariant Keypoints'').

    \begin{Verbatim}[commandchars=\\\{\}]
{\color{incolor}In [{\color{incolor}6}]:} \PY{n}{df}\PY{o}{.}\PY{n}{sort\PYZus{}values}\PY{p}{(}\PY{l+s+s2}{\PYZdq{}}\PY{l+s+s2}{flow}\PY{l+s+s2}{\PYZdq{}}\PY{p}{,} \PY{n}{ascending}\PY{o}{=}\PY{k+kc}{False}\PY{p}{)}\PY{p}{[}\PY{p}{[}\PY{l+s+s2}{\PYZdq{}}\PY{l+s+s2}{year}\PY{l+s+s2}{\PYZdq{}}\PY{p}{,}\PY{l+s+s2}{\PYZdq{}}\PY{l+s+s2}{title}\PY{l+s+s2}{\PYZdq{}}\PY{p}{,}\PY{l+s+s2}{\PYZdq{}}\PY{l+s+s2}{flow}\PY{l+s+s2}{\PYZdq{}}\PY{p}{]}\PY{p}{]}\PY{o}{.}\PY{n}{head}\PY{p}{(}\PY{p}{)}
\end{Verbatim}

            \begin{Verbatim}[commandchars=\\\{\}]
{\color{outcolor}Out[{\color{outcolor}6}]:}                                       year  \textbackslash{}
        id                                           
        6a6b9aa6-683f-4c7c-b06e-9c3018d10fd3  1989   
        b944f77f-113b-4a02-ae5e-d4a124b8fd5b  2004   
        c1b6b493-01ef-420f-be44-7bacfe34e846  2011   
        a662a4e7-415e-417e-8a8f-fe085d7e487f  1974   
        d3e00e7e-1c64-4d7a-b2b2-1ad98ba4c706  1988   
        
                                                                                          title  \textbackslash{}
        id                                                                                        
        6a6b9aa6-683f-4c7c-b06e-9c3018d10fd3  Genetic Algorithms in Search, Optimization and{\ldots}   
        b944f77f-113b-4a02-ae5e-d4a124b8fd5b  Distinctive Image Features from Scale-Invarian{\ldots}   
        c1b6b493-01ef-420f-be44-7bacfe34e846      LIBSVM: A library for support vector machines   
        a662a4e7-415e-417e-8a8f-fe085d7e487f     The Design and Analysis of Computer Algorithms   
        d3e00e7e-1c64-4d7a-b2b2-1ad98ba4c706  Probabilistic Reasoning in Intelligent Systems{\ldots}   
        
                                                  flow  
        id                                              
        6a6b9aa6-683f-4c7c-b06e-9c3018d10fd3  0.000735  
        b944f77f-113b-4a02-ae5e-d4a124b8fd5b  0.000599  
        c1b6b493-01ef-420f-be44-7bacfe34e846  0.000474  
        a662a4e7-415e-417e-8a8f-fe085d7e487f  0.000416  
        d3e00e7e-1c64-4d7a-b2b2-1ad98ba4c706  0.000372  
\end{Verbatim}
        
    \begin{Verbatim}[commandchars=\\\{\}]
{\color{incolor}In [{\color{incolor}57}]:} \PY{n}{df}\PY{o}{.}\PY{n}{flow}\PY{o}{.}\PY{n}{plot}\PY{p}{(}\PY{n}{kind}\PY{o}{=}\PY{l+s+s2}{\PYZdq{}}\PY{l+s+s2}{hist}\PY{l+s+s2}{\PYZdq{}}\PY{p}{,} \PY{n}{logy}\PY{o}{=}\PY{k+kc}{True}\PY{p}{,} \PY{n}{title}\PY{o}{=}\PY{l+s+s2}{\PYZdq{}}\PY{l+s+s2}{Histogram of Eigenfactor Scores}\PY{l+s+s2}{\PYZdq{}}\PY{p}{)}
\end{Verbatim}

            \begin{Verbatim}[commandchars=\\\{\}]
{\color{outcolor}Out[{\color{outcolor}57}]:} <matplotlib.axes.\_subplots.AxesSubplot at 0x262c5f710>
\end{Verbatim}
        
    \begin{center}
    \adjustimage{max size={0.9\linewidth}{0.9\paperheight}}{output_7_1.png}
    \end{center}
    { \hspace*{\fill} \\}
    
    Looking at the top papers by flow we find reasonable titles. Notably
these all seem to be real papers, and not mislabled journals or other
erros. The same holds true for in-citations.

    \begin{Verbatim}[commandchars=\\\{\}]
{\color{incolor}In [{\color{incolor}7}]:} \PY{n}{df}\PY{o}{.}\PY{n}{sort\PYZus{}values}\PY{p}{(}\PY{l+s+s2}{\PYZdq{}}\PY{l+s+s2}{in\PYZus{}cite}\PY{l+s+s2}{\PYZdq{}}\PY{p}{,} \PY{n}{ascending}\PY{o}{=}\PY{k+kc}{False}\PY{p}{)}\PY{p}{[}\PY{p}{[}\PY{l+s+s2}{\PYZdq{}}\PY{l+s+s2}{year}\PY{l+s+s2}{\PYZdq{}}\PY{p}{,}\PY{l+s+s2}{\PYZdq{}}\PY{l+s+s2}{title}\PY{l+s+s2}{\PYZdq{}}\PY{p}{,}\PY{l+s+s2}{\PYZdq{}}\PY{l+s+s2}{in\PYZus{}cite}\PY{l+s+s2}{\PYZdq{}}\PY{p}{]}\PY{p}{]}\PY{o}{.}\PY{n}{head}\PY{p}{(}\PY{p}{)}
\end{Verbatim}

            \begin{Verbatim}[commandchars=\\\{\}]
{\color{outcolor}Out[{\color{outcolor}7}]:}                                       year  \textbackslash{}
        id                                           
        b944f77f-113b-4a02-ae5e-d4a124b8fd5b  2004   
        c1b6b493-01ef-420f-be44-7bacfe34e846  2011   
        6a6b9aa6-683f-4c7c-b06e-9c3018d10fd3  1989   
        dd83785a-dd19-41e3-9b25-ebabbd48d336  2005   
        f6bd8b64-684d-429a-aab5-8ff3a2c23cd6  2001   
        
                                                                                          title  \textbackslash{}
        id                                                                                        
        b944f77f-113b-4a02-ae5e-d4a124b8fd5b  Distinctive Image Features from Scale-Invarian{\ldots}   
        c1b6b493-01ef-420f-be44-7bacfe34e846      LIBSVM: A library for support vector machines   
        6a6b9aa6-683f-4c7c-b06e-9c3018d10fd3  Genetic Algorithms in Search, Optimization and{\ldots}   
        dd83785a-dd19-41e3-9b25-ebabbd48d336  Histograms of oriented gradients for human det{\ldots}   
        f6bd8b64-684d-429a-aab5-8ff3a2c23cd6                                     Random Forests   
        
                                              in\_cite  
        id                                             
        b944f77f-113b-4a02-ae5e-d4a124b8fd5b  16229.0  
        c1b6b493-01ef-420f-be44-7bacfe34e846  13475.0  
        6a6b9aa6-683f-4c7c-b06e-9c3018d10fd3  13267.0  
        dd83785a-dd19-41e3-9b25-ebabbd48d336   8477.0  
        f6bd8b64-684d-429a-aab5-8ff3a2c23cd6   7968.0  
\end{Verbatim}
        
    \begin{Verbatim}[commandchars=\\\{\}]
{\color{incolor}In [{\color{incolor}12}]:} \PY{n}{df}\PY{o}{.}\PY{n}{in\PYZus{}cite}\PY{o}{.}\PY{n}{plot}\PY{p}{(}\PY{n}{kind}\PY{o}{=}\PY{l+s+s2}{\PYZdq{}}\PY{l+s+s2}{hist}\PY{l+s+s2}{\PYZdq{}}\PY{p}{,} \PY{n}{logy}\PY{o}{=}\PY{k+kc}{True}\PY{p}{,} \PY{n}{title}\PY{o}{=}\PY{l+s+s2}{\PYZdq{}}\PY{l+s+s2}{Histogram of in\PYZhy{}citations}\PY{l+s+s2}{\PYZdq{}}\PY{p}{)}
\end{Verbatim}

            \begin{Verbatim}[commandchars=\\\{\}]
{\color{outcolor}Out[{\color{outcolor}12}]:} <matplotlib.axes.\_subplots.AxesSubplot at 0x1deb5e550>
\end{Verbatim}
        
    \begin{center}
    \adjustimage{max size={0.9\linewidth}{0.9\paperheight}}{output_10_1.png}
    \end{center}
    { \hspace*{\fill} \\}
    
    \begin{Verbatim}[commandchars=\\\{\}]
{\color{incolor}In [{\color{incolor}8}]:} \PY{n}{df}\PY{o}{.}\PY{n}{sort\PYZus{}values}\PY{p}{(}\PY{l+s+s2}{\PYZdq{}}\PY{l+s+s2}{out\PYZus{}cite}\PY{l+s+s2}{\PYZdq{}}\PY{p}{,} \PY{n}{ascending}\PY{o}{=}\PY{k+kc}{False}\PY{p}{)}\PY{p}{[}\PY{p}{[}\PY{l+s+s2}{\PYZdq{}}\PY{l+s+s2}{year}\PY{l+s+s2}{\PYZdq{}}\PY{p}{,}\PY{l+s+s2}{\PYZdq{}}\PY{l+s+s2}{title}\PY{l+s+s2}{\PYZdq{}}\PY{p}{,}\PY{l+s+s2}{\PYZdq{}}\PY{l+s+s2}{out\PYZus{}cite}\PY{l+s+s2}{\PYZdq{}}\PY{p}{]}\PY{p}{]}\PY{o}{.}\PY{n}{head}\PY{p}{(}\PY{p}{)}
\end{Verbatim}

            \begin{Verbatim}[commandchars=\\\{\}]
{\color{outcolor}Out[{\color{outcolor}8}]:}                                       year  \textbackslash{}
        id                                           
        da170252-5470-4ed6-947f-04003976f579  1999   
        ac2acfac-4597-4ebd-b1c5-08c888d73271  1997   
        d3be0271-593b-44e9-a2c4-cacb26dc1833  2000   
        c6090c87-4730-4ae6-86cd-7663adb23b2b  2002   
        d0c11ccc-e743-4a45-bbbc-ba5030b78e33  1991   
        
                                                                                          title  \textbackslash{}
        id                                                                                        
        da170252-5470-4ed6-947f-04003976f579          Image analysis and computer vision : 1998   
        ac2acfac-4597-4ebd-b1c5-08c888d73271           Image analysis and computer vision: 1996   
        d3be0271-593b-44e9-a2c4-cacb26dc1833           Image analysis and computer vision: 1999   
        c6090c87-4730-4ae6-86cd-7663adb23b2b  Comprehensive frequency-dependent substrate no{\ldots}   
        d0c11ccc-e743-4a45-bbbc-ba5030b78e33           Image analysis and computer vision: 1990   
        
                                              out\_cite  
        id                                              
        da170252-5470-4ed6-947f-04003976f579      1532  
        ac2acfac-4597-4ebd-b1c5-08c888d73271      1362  
        d3be0271-593b-44e9-a2c4-cacb26dc1833      1001  
        c6090c87-4730-4ae6-86cd-7663adb23b2b       759  
        d0c11ccc-e743-4a45-bbbc-ba5030b78e33       734  
\end{Verbatim}
        
    \begin{Verbatim}[commandchars=\\\{\}]
{\color{incolor}In [{\color{incolor}58}]:} \PY{n}{df}\PY{o}{.}\PY{n}{out\PYZus{}cite}\PY{o}{.}\PY{n}{plot}\PY{p}{(}\PY{n}{kind}\PY{o}{=}\PY{l+s+s2}{\PYZdq{}}\PY{l+s+s2}{hist}\PY{l+s+s2}{\PYZdq{}}\PY{p}{,} \PY{n}{logy}\PY{o}{=}\PY{k+kc}{True}\PY{p}{,} \PY{n}{title}\PY{o}{=}\PY{l+s+s2}{\PYZdq{}}\PY{l+s+s2}{Histogram of out\PYZhy{}citations}\PY{l+s+s2}{\PYZdq{}}\PY{p}{)}
\end{Verbatim}

            \begin{Verbatim}[commandchars=\\\{\}]
{\color{outcolor}Out[{\color{outcolor}58}]:} <matplotlib.axes.\_subplots.AxesSubplot at 0x3270f09b0>
\end{Verbatim}
        
    \begin{center}
    \adjustimage{max size={0.9\linewidth}{0.9\paperheight}}{output_12_1.png}
    \end{center}
    { \hspace*{\fill} \\}
    
    Here we find an error,
\texttt{Image\ analysis\ and\ computer\ vision:\ 1998} is likely the
title of a journal, not an individual paper. This sort of error is
common in bibliometric datasets. A classic example is the Google Scholar
profile for the well known
\href{https://scholar.google.nl/citations?user=qGuYgMsAAAAJ\&hl=en}{et
al}, who has nearly 115k citations this year alone.

    \begin{Verbatim}[commandchars=\\\{\}]
{\color{incolor}In [{\color{incolor}15}]:} \PY{n}{df}\PY{o}{.}\PY{n}{sort\PYZus{}values}\PY{p}{(}\PY{l+s+s2}{\PYZdq{}}\PY{l+s+s2}{author\PYZus{}cnt}\PY{l+s+s2}{\PYZdq{}}\PY{p}{,} \PY{n}{ascending}\PY{o}{=}\PY{k+kc}{False}\PY{p}{)}\PY{p}{[}\PY{p}{[}\PY{l+s+s2}{\PYZdq{}}\PY{l+s+s2}{year}\PY{l+s+s2}{\PYZdq{}}\PY{p}{,}\PY{l+s+s2}{\PYZdq{}}\PY{l+s+s2}{title}\PY{l+s+s2}{\PYZdq{}}\PY{p}{,}\PY{l+s+s2}{\PYZdq{}}\PY{l+s+s2}{author\PYZus{}cnt}\PY{l+s+s2}{\PYZdq{}}\PY{p}{]}\PY{p}{]}\PY{o}{.}\PY{n}{head}\PY{p}{(}\PY{p}{)}
\end{Verbatim}

            \begin{Verbatim}[commandchars=\\\{\}]
{\color{outcolor}Out[{\color{outcolor}15}]:}                                       year  \textbackslash{}
         id                                           
         5de90b6c-92cb-40f5-898a-5eecd10c3d14  2017   
         23d6baa9-65fb-4cd1-a470-63b4227f2955  2015   
         2c12b8a7-4258-4513-b40a-475427d492c5  2014   
         66fe9d26-426c-4c98-ab7e-fe71dd4d48ed  2006   
         106b4db9-2ae6-4744-9533-f7809345f1ca  2002   
         
                                                                                           title  \textbackslash{}
         id                                                                                        
         5de90b6c-92cb-40f5-898a-5eecd10c3d14  Construction and Analysis of Weighted Brain Ne{\ldots}   
         23d6baa9-65fb-4cd1-a470-63b4227f2955  The IceProd framework : Distributed data proce{\ldots}   
         2c12b8a7-4258-4513-b40a-475427d492c5        A promoter-level mammalian expression atlas   
         66fe9d26-426c-4c98-ab7e-fe71dd4d48ed  Length Sensing and Control in the Virgo Gravit{\ldots}   
         106b4db9-2ae6-4744-9533-f7809345f1ca        An Overview of the BlueGene/L Supercomputer   
         
                                               author\_cnt  
         id                                                
         5de90b6c-92cb-40f5-898a-5eecd10c3d14         351  
         23d6baa9-65fb-4cd1-a470-63b4227f2955         286  
         2c12b8a7-4258-4513-b40a-475427d492c5         261  
         66fe9d26-426c-4c98-ab7e-fe71dd4d48ed         119  
         106b4db9-2ae6-4744-9533-f7809345f1ca         115  
\end{Verbatim}
        
    \begin{Verbatim}[commandchars=\\\{\}]
{\color{incolor}In [{\color{incolor}16}]:} \PY{n}{df}\PY{o}{.}\PY{n}{author\PYZus{}cnt}\PY{o}{.}\PY{n}{plot}\PY{p}{(}\PY{n}{kind}\PY{o}{=}\PY{l+s+s2}{\PYZdq{}}\PY{l+s+s2}{hist}\PY{l+s+s2}{\PYZdq{}}\PY{p}{,} \PY{n}{logy}\PY{o}{=}\PY{k+kc}{True}\PY{p}{,} \PY{n}{title}\PY{o}{=}\PY{l+s+s2}{\PYZdq{}}\PY{l+s+s2}{Histogram of author counts}\PY{l+s+s2}{\PYZdq{}}\PY{p}{)}
\end{Verbatim}

            \begin{Verbatim}[commandchars=\\\{\}]
{\color{outcolor}Out[{\color{outcolor}16}]:} <matplotlib.axes.\_subplots.AxesSubplot at 0x23a388f28>
\end{Verbatim}
        
    \begin{center}
    \adjustimage{max size={0.9\linewidth}{0.9\paperheight}}{output_15_1.png}
    \end{center}
    { \hspace*{\fill} \\}
    
    \begin{Verbatim}[commandchars=\\\{\}]
{\color{incolor}In [{\color{incolor}61}]:} \PY{n}{sns}\PY{o}{.}\PY{n}{pairplot}\PY{p}{(}\PY{n}{df}\PY{o}{.}\PY{n}{dropna}\PY{p}{(}\PY{p}{)}\PY{p}{,} \PY{n+nb}{vars}\PY{o}{=}\PY{p}{[}\PY{l+s+s2}{\PYZdq{}}\PY{l+s+s2}{in\PYZus{}cite}\PY{l+s+s2}{\PYZdq{}}\PY{p}{,}\PY{l+s+s2}{\PYZdq{}}\PY{l+s+s2}{out\PYZus{}cite}\PY{l+s+s2}{\PYZdq{}}\PY{p}{,}\PY{l+s+s2}{\PYZdq{}}\PY{l+s+s2}{flow}\PY{l+s+s2}{\PYZdq{}}\PY{p}{,} \PY{l+s+s2}{\PYZdq{}}\PY{l+s+s2}{year}\PY{l+s+s2}{\PYZdq{}}\PY{p}{]}\PY{p}{)}
\end{Verbatim}

            \begin{Verbatim}[commandchars=\\\{\}]
{\color{outcolor}Out[{\color{outcolor}61}]:} <seaborn.axisgrid.PairGrid at 0x220a4f550>
\end{Verbatim}
        
    \begin{center}
    \adjustimage{max size={0.9\linewidth}{0.9\paperheight}}{output_16_1.png}
    \end{center}
    { \hspace*{\fill} \\}
    
    \begin{Verbatim}[commandchars=\\\{\}]
{\color{incolor}In [{\color{incolor}40}]:} \PY{n}{df}\PY{o}{.}\PY{n}{groupby}\PY{p}{(}\PY{l+s+s2}{\PYZdq{}}\PY{l+s+s2}{year}\PY{l+s+s2}{\PYZdq{}}\PY{p}{)}\PY{o}{.}\PY{n}{size}\PY{p}{(}\PY{p}{)}\PY{o}{.}\PY{n}{plot}\PY{p}{(}\PY{n}{title}\PY{o}{=}\PY{l+s+s2}{\PYZdq{}}\PY{l+s+s2}{Number of papers by year}\PY{l+s+s2}{\PYZdq{}}\PY{p}{,} \PY{n}{figsize}\PY{o}{=}\PY{n}{figsize}\PY{p}{)}
\end{Verbatim}

            \begin{Verbatim}[commandchars=\\\{\}]
{\color{outcolor}Out[{\color{outcolor}40}]:} <matplotlib.axes.\_subplots.AxesSubplot at 0x1e601a908>
\end{Verbatim}
        
    \begin{center}
    \adjustimage{max size={0.9\linewidth}{0.9\paperheight}}{output_17_1.png}
    \end{center}
    { \hspace*{\fill} \\}
    
    \begin{Verbatim}[commandchars=\\\{\}]
{\color{incolor}In [{\color{incolor}76}]:} \PY{n}{df}\PY{p}{[}\PY{p}{[}\PY{l+s+s2}{\PYZdq{}}\PY{l+s+s2}{in\PYZus{}cite}\PY{l+s+s2}{\PYZdq{}}\PY{p}{,} \PY{l+s+s2}{\PYZdq{}}\PY{l+s+s2}{out\PYZus{}cite}\PY{l+s+s2}{\PYZdq{}}\PY{p}{,} \PY{l+s+s2}{\PYZdq{}}\PY{l+s+s2}{year}\PY{l+s+s2}{\PYZdq{}}\PY{p}{]}\PY{p}{]}\PY{o}{.}\PY{n}{groupby}\PY{p}{(}\PY{l+s+s2}{\PYZdq{}}\PY{l+s+s2}{year}\PY{l+s+s2}{\PYZdq{}}\PY{p}{)}\PY{o}{.}\PY{n}{mean}\PY{p}{(}\PY{p}{)}\PY{o}{.}\PY{n}{plot}\PY{p}{(}\PY{n}{logy}\PY{o}{=}\PY{k+kc}{True}\PY{p}{,} \PY{n}{figsize}\PY{o}{=}\PY{n}{figsize}\PY{p}{,} \PY{n}{title}\PY{o}{=}\PY{l+s+s2}{\PYZdq{}}\PY{l+s+s2}{Average citation counts over time}\PY{l+s+s2}{\PYZdq{}}\PY{p}{)}
\end{Verbatim}

            \begin{Verbatim}[commandchars=\\\{\}]
{\color{outcolor}Out[{\color{outcolor}76}]:} <matplotlib.axes.\_subplots.AxesSubplot at 0x265fcea90>
\end{Verbatim}
        
    \begin{center}
    \adjustimage{max size={0.9\linewidth}{0.9\paperheight}}{output_18_1.png}
    \end{center}
    { \hspace*{\fill} \\}
    
    Since we are investigating questions around citation patterns we should
understand what they look like in this dataset. First, the top plot
shows a sharp increase in the number of papers after 2000. In the second
plot we see a steady increase in the average number of out citations
over time, while the average in bound citations remain steady until
apporximately 2000, where they begin falling off. This could be due to
papers not yet being fixed in the citation graph, though 10 years is a
longer estimate than is normally provided, especially for copmuter
science, which this collection covers.

    \begin{Verbatim}[commandchars=\\\{\}]
{\color{incolor}In [{\color{incolor}68}]:} \PY{k}{def} \PY{n+nf}{lorenz\PYZus{}curve}\PY{p}{(}\PY{n}{ds}\PY{p}{,} \PY{n}{quantiles}\PY{o}{=}\PY{p}{[}\PY{o}{.}\PY{l+m+mi}{2}\PY{p}{,}\PY{o}{.}\PY{l+m+mi}{4}\PY{p}{,}\PY{o}{.}\PY{l+m+mi}{6}\PY{p}{,}\PY{o}{.}\PY{l+m+mi}{8}\PY{p}{,}\PY{l+m+mi}{1}\PY{p}{]}\PY{p}{)}\PY{p}{:}
             \PY{n}{breaks} \PY{o}{=} \PY{n}{ds}\PY{o}{.}\PY{n}{quantile}\PY{p}{(}\PY{n}{quantiles}\PY{p}{)}
             \PY{n}{t} \PY{o}{=} \PY{n}{ds}\PY{o}{.}\PY{n}{sum}\PY{p}{(}\PY{p}{)}
             \PY{n}{cs} \PY{o}{=} \PY{p}{[}\PY{n}{ds}\PY{p}{[}\PY{n}{ds} \PY{o}{\PYZlt{}}\PY{o}{=} \PY{n}{b}\PY{p}{]}\PY{o}{.}\PY{n}{sum}\PY{p}{(}\PY{p}{)}\PY{o}{/}\PY{n}{t} \PY{k}{for} \PY{n}{b} \PY{o+ow}{in} \PY{n}{breaks}\PY{p}{]}
             \PY{n}{cs}\PY{o}{.}\PY{n}{insert}\PY{p}{(}\PY{l+m+mi}{0}\PY{p}{,} \PY{l+m+mi}{0}\PY{p}{)}
             \PY{n}{quantiles}\PY{o}{.}\PY{n}{insert}\PY{p}{(}\PY{l+m+mi}{0}\PY{p}{,}\PY{l+m+mi}{0}\PY{p}{)}
             \PY{k}{return} \PY{n}{pd}\PY{o}{.}\PY{n}{Series}\PY{p}{(}\PY{n}{cs}\PY{p}{,} \PY{n}{index}\PY{o}{=}\PY{n}{quantiles}\PY{p}{)}
         
         \PY{k}{def} \PY{n+nf}{plot\PYZus{}lorenz}\PY{p}{(}\PY{n}{ls}\PY{p}{,} \PY{n}{tl}\PY{o}{=}\PY{k+kc}{None}\PY{p}{)}\PY{p}{:}
             \PY{n}{ax} \PY{o}{=} \PY{n}{ls}\PY{o}{.}\PY{n}{plot}\PY{p}{(}\PY{n}{title}\PY{o}{=}\PY{n}{tl}\PY{p}{)}
             \PY{n}{ax}\PY{o}{.}\PY{n}{plot}\PY{p}{(}\PY{p}{[}\PY{l+m+mi}{0}\PY{p}{,}\PY{l+m+mi}{1}\PY{p}{]}\PY{p}{,}\PY{p}{[}\PY{l+m+mi}{0}\PY{p}{,}\PY{l+m+mi}{1}\PY{p}{]}\PY{p}{)}
\end{Verbatim}

    \begin{Verbatim}[commandchars=\\\{\}]
{\color{incolor}In [{\color{incolor}71}]:} \PY{n}{plot\PYZus{}lorenz}\PY{p}{(}\PY{n}{lorenz\PYZus{}curve}\PY{p}{(}\PY{n}{df}\PY{o}{.}\PY{n}{in\PYZus{}cite}\PY{p}{)}\PY{p}{,} \PY{l+s+s2}{\PYZdq{}}\PY{l+s+s2}{Lorenz Curve for in\PYZhy{}citations}\PY{l+s+s2}{\PYZdq{}}\PY{p}{)}
\end{Verbatim}

    \begin{center}
    \adjustimage{max size={0.9\linewidth}{0.9\paperheight}}{output_21_0.png}
    \end{center}
    { \hspace*{\fill} \\}
    
    \begin{Verbatim}[commandchars=\\\{\}]
{\color{incolor}In [{\color{incolor}72}]:} \PY{n}{plot\PYZus{}lorenz}\PY{p}{(}\PY{n}{lorenz\PYZus{}curve}\PY{p}{(}\PY{n}{df}\PY{o}{.}\PY{n}{flow}\PY{p}{)}\PY{p}{,} \PY{l+s+s2}{\PYZdq{}}\PY{l+s+s2}{Lorenz Curve for flow}\PY{l+s+s2}{\PYZdq{}}\PY{p}{)}
\end{Verbatim}

    \begin{center}
    \adjustimage{max size={0.9\linewidth}{0.9\paperheight}}{output_22_0.png}
    \end{center}
    { \hspace*{\fill} \\}
    
    \begin{Verbatim}[commandchars=\\\{\}]
{\color{incolor}In [{\color{incolor}73}]:} \PY{n}{plot\PYZus{}lorenz}\PY{p}{(}\PY{n}{lorenz\PYZus{}curve}\PY{p}{(}\PY{n}{df}\PY{o}{.}\PY{n}{out\PYZus{}cite}\PY{p}{)}\PY{p}{,} \PY{l+s+s2}{\PYZdq{}}\PY{l+s+s2}{Lorenz Curve for out\PYZhy{}citations}\PY{l+s+s2}{\PYZdq{}}\PY{p}{)}
\end{Verbatim}

    \begin{center}
    \adjustimage{max size={0.9\linewidth}{0.9\paperheight}}{output_23_0.png}
    \end{center}
    { \hspace*{\fill} \\}
    
    In the above plots we compare the Lorenz Curves for in-citations,
Eigenfactor score (flow), and out-citations. A Lorenz Curve is a
visualization of the distribution of some value, with the x-axis being
the cumulative percent of the populace (papers in this case), and the
y-axis the cumulative percent of the value being examined. The Yellow
line in each represents perfect equality (e.g.~each paper has the exact
same number of citations). It is interesting to see that Eigenfactor
scores have worse inequality than in-citations.

    A common method for calculating inequality is via the Gini coefficient.
We calculate the Gini coefficient for a non-decreasing
\(y_i <= y_{i+1}\) as:
\[G = \frac{1}{n}(n+1-2\frac{\sum_{i=1}^{n} (n+1-i)y_i}{\sum_{i=1}^{n}y_i})\]

    \begin{Verbatim}[commandchars=\\\{\}]
{\color{incolor}In [{\color{incolor}21}]:} \PY{k}{def} \PY{n+nf}{gini}\PY{p}{(}\PY{n}{vs}\PY{p}{)}\PY{p}{:}
             \PY{k}{if} \PY{n+nb}{isinstance}\PY{p}{(}\PY{n}{vs}\PY{p}{,} \PY{n}{pd}\PY{o}{.}\PY{n}{Series}\PY{p}{)}\PY{p}{:}
                 \PY{n}{vs} \PY{o}{=} \PY{n}{vs}\PY{o}{.}\PY{n}{values}
             \PY{n}{vs} \PY{o}{=} \PY{n}{vs}\PY{p}{[}\PY{o}{\PYZti{}}\PY{n}{np}\PY{o}{.}\PY{n}{isnan}\PY{p}{(}\PY{n}{vs}\PY{p}{)}\PY{p}{]}
             \PY{n}{s} \PY{o}{=} \PY{l+m+mi}{0}
             \PY{n}{n} \PY{o}{=} \PY{n+nb}{len}\PY{p}{(}\PY{n}{vs}\PY{p}{)}
             \PY{n}{vs}\PY{o}{.}\PY{n}{sort}\PY{p}{(}\PY{p}{)}
             \PY{n}{svs} \PY{o}{=} \PY{n+nb}{sum}\PY{p}{(}\PY{n}{vs}\PY{p}{)}
             \PY{k}{if} \PY{n}{svs} \PY{o}{==} \PY{l+m+mi}{0}\PY{p}{:}
                 \PY{k}{return} \PY{l+m+mi}{0}
             \PY{k}{for} \PY{n}{i}\PY{p}{,} \PY{n}{y} \PY{o+ow}{in} \PY{n+nb}{enumerate}\PY{p}{(}\PY{n}{vs}\PY{p}{,} \PY{n}{start}\PY{o}{=}\PY{l+m+mi}{1}\PY{p}{)}\PY{p}{:}
                 \PY{n}{s}\PY{o}{+}\PY{o}{=}\PY{p}{(}\PY{n}{n}\PY{o}{+}\PY{l+m+mi}{1}\PY{o}{\PYZhy{}}\PY{n}{i}\PY{p}{)}\PY{o}{*}\PY{n}{y}
             \PY{k}{return} \PY{l+m+mi}{1}\PY{o}{/}\PY{n}{n}\PY{o}{*}\PY{p}{(}\PY{n}{n}\PY{o}{+}\PY{l+m+mi}{1}\PY{o}{\PYZhy{}}\PY{l+m+mi}{2}\PY{o}{*}\PY{n}{s}\PY{o}{/}\PY{n}{svs}\PY{p}{)}
\end{Verbatim}

    \begin{Verbatim}[commandchars=\\\{\}]
{\color{incolor}In [{\color{incolor}75}]:} \PY{n}{gini}\PY{p}{(}\PY{n}{df}\PY{o}{.}\PY{n}{in\PYZus{}cite}\PY{p}{)}
\end{Verbatim}

            \begin{Verbatim}[commandchars=\\\{\}]
{\color{outcolor}Out[{\color{outcolor}75}]:} 0.71622795538327888
\end{Verbatim}
        
    Above we calculate the Gini coefficient over the entire dataset for
in-citations. We find an answer of 71.6. If we were looking at global
income inequality this would be higher than any country in the world.
The World Bank estimates South Africa's coeficient at 63.4. For
reference, the US has a coeficient of 41. This means the distribution of
in-citations in the DBLP corpus is more unequal than the most unequal
country.

    \begin{Verbatim}[commandchars=\\\{\}]
{\color{incolor}In [{\color{incolor}45}]:} \PY{n}{gi} \PY{o}{=} \PY{n}{df}\PY{p}{[}\PY{p}{[}\PY{l+s+s2}{\PYZdq{}}\PY{l+s+s2}{year}\PY{l+s+s2}{\PYZdq{}}\PY{p}{,}\PY{l+s+s2}{\PYZdq{}}\PY{l+s+s2}{in\PYZus{}cite}\PY{l+s+s2}{\PYZdq{}}\PY{p}{]}\PY{p}{]}\PY{o}{.}\PY{n}{groupby}\PY{p}{(}\PY{l+s+s2}{\PYZdq{}}\PY{l+s+s2}{year}\PY{l+s+s2}{\PYZdq{}}\PY{p}{)}\PY{o}{.}\PY{n}{agg}\PY{p}{(}\PY{n}{gini}\PY{p}{)}
         \PY{n}{gf} \PY{o}{=} \PY{n}{df}\PY{p}{[}\PY{p}{[}\PY{l+s+s2}{\PYZdq{}}\PY{l+s+s2}{year}\PY{l+s+s2}{\PYZdq{}}\PY{p}{,}\PY{l+s+s2}{\PYZdq{}}\PY{l+s+s2}{flow}\PY{l+s+s2}{\PYZdq{}}\PY{p}{]}\PY{p}{]}\PY{o}{.}\PY{n}{groupby}\PY{p}{(}\PY{l+s+s2}{\PYZdq{}}\PY{l+s+s2}{year}\PY{l+s+s2}{\PYZdq{}}\PY{p}{)}\PY{o}{.}\PY{n}{agg}\PY{p}{(}\PY{n}{gini}\PY{p}{)}
         \PY{n}{gi}\PY{o}{.}\PY{n}{join}\PY{p}{(}\PY{n}{gf}\PY{p}{)}\PY{o}{.}\PY{n}{plot}\PY{p}{(}\PY{n}{title}\PY{o}{=}\PY{l+s+s2}{\PYZdq{}}\PY{l+s+s2}{Gini coefficient for flow and in\PYZhy{}citations by year}\PY{l+s+s2}{\PYZdq{}}\PY{p}{,} \PY{n}{figsize}\PY{o}{=}\PY{n}{figsize}\PY{p}{)}
\end{Verbatim}

            \begin{Verbatim}[commandchars=\\\{\}]
{\color{outcolor}Out[{\color{outcolor}45}]:} <matplotlib.axes.\_subplots.AxesSubplot at 0x239986898>
\end{Verbatim}
        
    \begin{center}
    \adjustimage{max size={0.9\linewidth}{0.9\paperheight}}{output_29_1.png}
    \end{center}
    { \hspace*{\fill} \\}
    
    Here we calcuate the Gini coefficient for in-citations and Eigenfactor
score (flow) for each year. There appears to be a slight decrease from
the 1970's on, with a drastic drop after 2000, corresponding to the drop
in the average in-citations we observed earlier. There are a few
possible interpretations here: 1) paper citations becoming more diverse
or 2) the highly cited/classic papers haven't yet been identified. Given
the corresponding drop in the average number of in-citations (1) seems
to have more evidence supporting it.

    \begin{Verbatim}[commandchars=\\\{\}]
{\color{incolor}In [{\color{incolor}49}]:} \PY{n}{df}\PY{p}{[}\PY{p}{[}\PY{l+s+s2}{\PYZdq{}}\PY{l+s+s2}{in\PYZus{}cite}\PY{l+s+s2}{\PYZdq{}}\PY{p}{,} \PY{l+s+s2}{\PYZdq{}}\PY{l+s+s2}{out\PYZus{}cite}\PY{l+s+s2}{\PYZdq{}}\PY{p}{,} \PY{l+s+s2}{\PYZdq{}}\PY{l+s+s2}{year}\PY{l+s+s2}{\PYZdq{}}\PY{p}{,} \PY{l+s+s2}{\PYZdq{}}\PY{l+s+s2}{author\PYZus{}cnt}\PY{l+s+s2}{\PYZdq{}}\PY{p}{,} \PY{l+s+s2}{\PYZdq{}}\PY{l+s+s2}{title\PYZus{}len}\PY{l+s+s2}{\PYZdq{}}\PY{p}{]}\PY{p}{]}\PY{o}{.}\PY{n}{corr}\PY{p}{(}\PY{p}{)}
\end{Verbatim}

            \begin{Verbatim}[commandchars=\\\{\}]
{\color{outcolor}Out[{\color{outcolor}49}]:}              in\_cite  out\_cite      year  author\_cnt  title\_len
         in\_cite     1.000000  0.082356 -0.102966   -0.006078  -0.053284
         out\_cite    0.082356  1.000000  0.221947    0.069297   0.000437
         year       -0.102966  0.221947  1.000000    0.230172   0.185475
         author\_cnt -0.006078  0.069297  0.230172    1.000000   0.129574
         title\_len  -0.053284  0.000437  0.185475    0.129574   1.000000
\end{Verbatim}
        
    If there is a change in citation counts over time we might find evidence
of this in correlations. We find that in-citations are weakly negatively
correlated with time, which makes sense: older papers have more time to
gather citations. We also see out-citatoins positively correlated with
time, a finding that matches earlier analysis.

    \begin{Verbatim}[commandchars=\\\{\}]
{\color{incolor}In [{\color{incolor}55}]:} \PY{n}{df}\PY{p}{[}\PY{p}{[}\PY{l+s+s2}{\PYZdq{}}\PY{l+s+s2}{title\PYZus{}len}\PY{l+s+s2}{\PYZdq{}}\PY{p}{,} \PY{l+s+s2}{\PYZdq{}}\PY{l+s+s2}{author\PYZus{}cnt}\PY{l+s+s2}{\PYZdq{}}\PY{p}{,} \PY{l+s+s2}{\PYZdq{}}\PY{l+s+s2}{year}\PY{l+s+s2}{\PYZdq{}}\PY{p}{]}\PY{p}{]}\PY{o}{.}\PY{n}{groupby}\PY{p}{(}\PY{l+s+s2}{\PYZdq{}}\PY{l+s+s2}{year}\PY{l+s+s2}{\PYZdq{}}\PY{p}{)}\PY{o}{.}\PY{n}{mean}\PY{p}{(}\PY{p}{)}\PY{o}{.}\PY{n}{plot}\PY{p}{(}\PY{n}{figsize}\PY{o}{=}\PY{n}{figsize}\PY{p}{,} \PY{n}{secondary\PYZus{}y}\PY{o}{=}\PY{p}{[}\PY{l+s+s2}{\PYZdq{}}\PY{l+s+s2}{author\PYZus{}cnt}\PY{l+s+s2}{\PYZdq{}}\PY{p}{]}\PY{p}{)}
\end{Verbatim}

            \begin{Verbatim}[commandchars=\\\{\}]
{\color{outcolor}Out[{\color{outcolor}55}]:} <matplotlib.axes.\_subplots.AxesSubplot at 0x1e54e0b70>
\end{Verbatim}
        
    \begin{center}
    \adjustimage{max size={0.9\linewidth}{0.9\paperheight}}{output_33_1.png}
    \end{center}
    { \hspace*{\fill} \\}
    
    Finally, for a bit of fun I decided to see if the number of authors or
title lengths are changing over time. It turns out both are increasing,
nearly doubling over the lifetime of the entire dataset. I can
understand the increase in authors, there are more scientists than there
ever have been. I am less clear on why title lengths have gotten so much
longer though, perhaps some kind of academic SEO in action?

    \begin{Verbatim}[commandchars=\\\{\}]
{\color{incolor}In [{\color{incolor}78}]:} \PY{n}{df}\PY{p}{[}\PY{p}{[}\PY{l+s+s2}{\PYZdq{}}\PY{l+s+s2}{in\PYZus{}cite}\PY{l+s+s2}{\PYZdq{}}\PY{p}{,} \PY{l+s+s2}{\PYZdq{}}\PY{l+s+s2}{out\PYZus{}cite}\PY{l+s+s2}{\PYZdq{}}\PY{p}{,} \PY{l+s+s2}{\PYZdq{}}\PY{l+s+s2}{year}\PY{l+s+s2}{\PYZdq{}}\PY{p}{]}\PY{p}{]}\PY{o}{.}\PY{n}{groupby}\PY{p}{(}\PY{l+s+s2}{\PYZdq{}}\PY{l+s+s2}{year}\PY{l+s+s2}{\PYZdq{}}\PY{p}{)}\PY{o}{.}\PY{n}{max}\PY{p}{(}\PY{p}{)}\PY{o}{.}\PY{n}{plot}\PY{p}{(}\PY{n}{logy}\PY{o}{=}\PY{k+kc}{True}\PY{p}{,} \PY{n}{figsize}\PY{o}{=}\PY{n}{figsize}\PY{p}{,} \PY{n}{title}\PY{o}{=}\PY{l+s+s2}{\PYZdq{}}\PY{l+s+s2}{Max citation counts over time}\PY{l+s+s2}{\PYZdq{}}\PY{p}{)}
\end{Verbatim}

            \begin{Verbatim}[commandchars=\\\{\}]
{\color{outcolor}Out[{\color{outcolor}78}]:} <matplotlib.axes.\_subplots.AxesSubplot at 0x3169cc710>
\end{Verbatim}
        
    \begin{center}
    \adjustimage{max size={0.9\linewidth}{0.9\paperheight}}{output_35_1.png}
    \end{center}
    { \hspace*{\fill} \\}
    
    \hypertarget{conclusions}{%
\subsection{Conclusions}\label{conclusions}}

I analyzed the DBLP computer science bibliography looking for effects of
recommenders in the citation patterns of the papers, specifically the
Matthew effect (the rich get richer, while the poor get poorer). Using a
Gini coefficient for each year I did not increasing inequality in
in-citations over time. Actually there appears to be a small decrease
over time. This could correspond to an increasing number of papers being
written, which are diluting the effects of superstars. This would be the
case if there were some sort of upper limit on in-citations. The above
graph investigates this, showing that the maximum number of in-citations
over time is increasing, though perhaps not as quickly as the number of
additional papers being written.

Future investigations should try and derive a distance metric for
papers. This would allow us to measure how diverse citations lists are
for each year.


    % Add a bibliography block to the postdoc
    
    
    
    \end{document}
